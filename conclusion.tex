\section{Conclusión}

Las páginas web descentralizadas están redefiniendo la interacción entre usuarios y blockchain, ofreciendo nuevas formas de autenticación, mayor privacidad y control de datos. A través de la implementación de identidades descentralizadas y el uso de wallets para autenticación, los usuarios pueden interactuar de manera segura y directa con aplicaciones y servicios sin depender de intermediarios.

Los avances en tecnologías de almacenamiento descentralizado y criptografía han permitido proteger los datos de manera más eficaz, aunque persisten desafíos técnicos en términos de usabilidad, escalabilidad y seguridad que deben ser abordados para facilitar la adopción masiva.

La capacidad de interactuar con contratos inteligentes desde el frontend abre posibilidades para aplicaciones más dinámicas y autónomas, pero requiere atención cuidadosa a los detalles de implementación y seguridad.

Mirando hacia el futuro, Web3 promete otorgar a los usuarios un control sin precedentes sobre su identidad y datos personales. Como profesionales de la ingeniería, es crucial continuar innovando y desarrollando soluciones que promuevan una internet más descentralizada, segura y centrada en el usuario.

Este es un momento emocionante en la evolución tecnológica, y las decisiones que tomemos ahora darán forma al paisaje digital de las próximas décadas. Es nuestra responsabilidad asegurarnos de que este futuro sea inclusivo, sostenible y beneficioso para todos.

