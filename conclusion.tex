\section{Conclusión}

Las páginas web descentralizadas representan un cambio significativo en la forma en que los usuarios interactúan con la tecnología blockchain, ofreciendo mejoras en seguridad, privacidad y control de datos. A través de la implementación de identidades descentralizadas, autenticación mediante wallets y la interacción directa con contratos inteligentes, se está construyendo un ecosistema que prioriza la autonomía del usuario.

Desde una perspectiva de ingeniería, estos avances requieren un profundo entendimiento de criptografía, sistemas distribuidos y diseño de interfaces de usuario. Los desafíos técnicos, como la escalabilidad, la usabilidad y la seguridad, deben ser abordados mediante soluciones innovadoras y colaborativas.

El futuro de la Web3 promete un entorno donde los individuos tienen mayor control sobre su identidad y datos personales, transformando la interacción en línea y potenciando nuevas oportunidades. La colaboración entre desarrolladores, reguladores y la comunidad es esencial para construir una internet más descentralizada y equitativa.

La adopción masiva de estas tecnologías dependerá de nuestra capacidad para crear sistemas seguros, eficientes y fáciles de usar. Es responsabilidad de la comunidad de ingeniería liderar este cambio, estableciendo estándares y mejores prácticas que permitan a todos beneficiarse de las ventajas de la descentralización.

