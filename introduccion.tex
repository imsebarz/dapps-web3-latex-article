\section{Introducción}

La creciente adopción de tecnologías descentralizadas y la aparición de la blockchain han impulsado una revolución en la forma en que se concibe y utiliza internet. Las páginas web descentralizadas (DWebs) representan una evolución significativa respecto a las arquitecturas tradicionales, ofreciendo una infraestructura que elimina intermediarios y centraliza el control en los usuarios. Este cambio paradigmático tiene implicaciones profundas en áreas como la autenticación, la privacidad y el control de datos.

Las DWebs se construyen sobre protocolos descentralizados y sistemas de almacenamiento distribuidos, como IPFS (InterPlanetary File System), que permiten una distribución más equitativa y resiliente del contenido. La integración de la blockchain en este contexto proporciona un nivel adicional de seguridad y transparencia, esencial para aplicaciones críticas en ingeniería y sistemas informáticos.

Este artículo explora las interacciones entre los usuarios y la blockchain en el ámbito de las DWebs, analizando cómo se redefinen los mecanismos de autenticación, cómo se protege la privacidad y cómo los usuarios obtienen un mayor control sobre sus datos. Además, se abordan los retos técnicos actuales y se vislumbra el futuro de estas tecnologías en el contexto de Web3.

Las tecnologías descentralizadas no solo modifican la infraestructura técnica de internet, sino que también impactan en aspectos sociales y económicos, promoviendo una mayor equidad y participación de los usuarios. En ingeniería, esto se traduce en nuevas oportunidades y desafíos en el diseño y desarrollo de sistemas más seguros, eficientes y centrados en el usuario.

