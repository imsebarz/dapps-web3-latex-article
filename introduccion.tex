\section{Introducción}

En los últimos años, la tecnología blockchain ha emergido como una revolución en la forma en que se gestionan y almacenan los datos \cite{nakamoto2008bitcoin}. Su naturaleza descentralizada, inmutable y transparente la convierte en una solución ideal para resolver problemas de confianza y seguridad en sistemas distribuidos. Las páginas web descentralizadas (DWebs) están cambiando radicalmente la interacción del usuario con la blockchain, proporcionando mayor seguridad y control sobre sus datos personales.

La arquitectura tradicional de las aplicaciones web sigue un modelo cliente-servidor, donde los datos y la lógica de negocio residen en servidores centralizados. Este enfoque presenta vulnerabilidades significativas, incluyendo puntos únicos de falla, riesgos de seguridad y problemas de privacidad. La adopción de DWebs y tecnologías blockchain permite superar estas limitaciones al distribuir la carga y el control entre los participantes de la red.

Este documento explora cómo la descentralización está redefiniendo la autenticación, la privacidad y el control de datos en el entorno web. Se abordarán los conceptos clave, las tecnologías involucradas y los desafíos técnicos asociados con esta transición, proporcionando una visión integral de las oportunidades y retos que presenta la Web3.

El objetivo principal es analizar cómo las DWebs pueden mejorar la experiencia del usuario al ofrecer mayor seguridad y autonomía, aspectos cada vez más valorados en la era digital. Además, se discutirá el impacto de estas tecnologías en la ingeniería de software y los sistemas de información, enfatizando las consideraciones técnicas y las mejores prácticas para su implementación.

