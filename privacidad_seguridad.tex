\section{Privacidad y Seguridad de Datos en Blockchain}

La blockchain, por su naturaleza descentralizada y cifrada, ofrece un entorno seguro para el almacenamiento y transferencia de datos \cite{bonneau2015sok}. En las DWebs, los datos del usuario no se almacenan en servidores centralizados sino en la propia blockchain o en sistemas de almacenamiento descentralizado como IPFS (InterPlanetary File System) \cite{benet2014ipfs} y Swarm \cite{swarm2016}.

La arquitectura de almacenamiento descentralizado utiliza técnicas de hashing y direccionamiento por contenido, donde los datos se dividen en fragmentos y se distribuyen a través de la red \cite{benet2014ipfs}. Los usuarios pueden recuperar los datos utilizando hashes criptográficos, garantizando la integridad y disponibilidad de la información sin depender de un solo proveedor \cite{wang2019survey}.

La privacidad se ve reforzada mediante el uso de criptografía avanzada, como el cifrado de extremo a extremo y las pruebas de conocimiento cero (ZKPs) \cite{ben2014zerocash}. Estas técnicas permiten realizar transacciones y operaciones sin revelar información sensible, protegiendo la identidad y los datos del usuario \cite{mohanty2018ethereum}.

Sin embargo, la transparencia inherente de la blockchain presenta desafíos para la privacidad, ya que las transacciones y los datos almacenados son accesibles públicamente \cite{meiklejohn2013fistful}. Para abordar esto, se están desarrollando soluciones como:

\begin{itemize}
    \item \textbf{Transacciones confidenciales}: Utilizan técnicas criptográficas para ocultar los detalles de las transacciones, como los montos y las direcciones involucradas \cite{maxwell2016confidential}.
    \item \textbf{Redes de capa dos}: Como Lightning Network, que permite transacciones fuera de la cadena (off-chain) con mayor privacidad y escalabilidad \cite{poon2016bitcoin}.
    \item \textbf{Monedas de privacidad}: Como Monero y Zcash, que implementan funciones específicas para mejorar el anonimato \cite{moser2018empirical}.
\end{itemize}

Desde una perspectiva de ingeniería, es crucial diseñar sistemas que equilibren la transparencia y la descentralización con la necesidad de privacidad del usuario. Esto implica implementar protocolos criptográficos avanzados, prácticas de seguridad robustas y considerar el cumplimiento con regulaciones como GDPR \cite{finck2018blockchain}.

Además, los ingenieros deben abordar cuestiones de rendimiento y eficiencia, ya que las operaciones criptográficas intensivas pueden afectar la usabilidad y la experiencia del usuario \cite{croman2016scaling}.

