\section{Interacción con Contratos Inteligentes desde el Frontend}

Los contratos inteligentes son programas que se ejecutan en la blockchain y permiten automatizar acuerdos y transacciones sin intermediarios. La interacción directa con estos contratos desde el frontend de una aplicación web es esencial para aprovechar plenamente las ventajas de las DWebs.

\subsection{APIs y Herramientas}

Para interactuar con contratos inteligentes desde el navegador, se utilizan bibliotecas y APIs como:

\begin{itemize}
    \item \textbf{Web3.js}: Una biblioteca de JavaScript que permite interactuar con la blockchain de Ethereum.
    \item \textbf{Ethers.js}: Una alternativa ligera a Web3.js con una API similar.
    \item \textbf{Truffle Suite}: Incluye herramientas para desarrollar, probar e interactuar con contratos inteligentes.
\end{itemize}

\subsection{Proceso de Interacción}

El flujo típico para interactuar con un contrato inteligente desde el frontend es:

\begin{enumerate}
    \item \textbf{Conexión a la Blockchain}: Establecer una conexión a un nodo de la red, ya sea localmente o a través de proveedores como Infura.
    \item \textbf{Instanciación del Contrato}: Utilizar la dirección y el ABI (Application Binary Interface) del contrato para crear una instancia en el frontend.
    \item \textbf{Lectura de Datos}: Llamar a funciones de solo lectura (\textit{view} o \textit{pure}) para obtener información sin costo de gas.
    \item \textbf{Envío de Transacciones}: Invocar funciones que modifican el estado del contrato, lo cual requiere una transacción firmada y el pago de gas.
\end{enumerate}

\subsection{Consideraciones de Seguridad}

La interacción con contratos inteligentes desde el frontend implica riesgos que deben ser gestionados:

\begin{itemize}
    \item \textbf{Validación de Datos}: Siempre validar la entrada del usuario antes de enviar transacciones para evitar ataques como inyección de código.
    \item \textbf{Gestión de Errores}: Implementar mecanismos para manejar errores y excepciones que puedan ocurrir durante la interacción con la blockchain.
    \item \textbf{Protección de Claves Privadas}: Las claves privadas nunca deben ser expuestas en el frontend; las wallets gestionan las firmas de forma segura.
\end{itemize}

\subsection{Experiencia del Usuario}

Para mejorar la UX, se pueden implementar:

\begin{itemize}
    \item \textbf{Notificaciones en Tiempo Real}: Informar al usuario sobre el estado de las transacciones.
    \item \textbf{Optimización de Gas}: Calcular y sugerir tarifas de gas óptimas para acelerar las transacciones.
    \item \textbf{Compatibilidad Multired}: Soportar múltiples redes (Mainnet, Testnets) y permitir al usuario seleccionar su preferencia.
\end{itemize}

\subsection{Ejemplo de Código}

A continuación, un ejemplo simplificado de cómo llamar a una función de un contrato inteligente desde el frontend usando Web3.js:

\begin{verbatim}
// Configurar Web3 y el contrato
const web3 = new Web3(window.ethereum);
const contract = new web3.eth.Contract(abi, contractAddress);

// Función para llamar a una función del contrato
async function callContractFunction() {
    const accounts = await web3.eth.getAccounts();
    const result = await contract.methods.myFunction().send({ from: accounts[0] });
    console.log('Resultado:', result);
}
\end{verbatim}

Este código asume que el usuario tiene una wallet como MetaMask instalada y que ha otorgado permiso a la aplicación para acceder a sus cuentas.

