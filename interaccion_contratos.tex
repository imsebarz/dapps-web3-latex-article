\section{Interacción con Contratos Inteligentes desde el Frontend}

Los contratos inteligentes son programas autoejecutables que se ejecutan en la blockchain y automatizan procesos sin intermediarios \cite{buterin2014next}. Las DWebs permiten a los usuarios interactuar directamente con estos contratos desde el navegador, lo que abre un abanico de posibilidades para el desarrollo de aplicaciones descentralizadas.

\subsection{Arquitectura de Interacción}

La interacción con contratos inteligentes desde el frontend implica una arquitectura cliente que se comunica directamente con la blockchain, sin la necesidad de servidores intermedios \cite{mavridou2018tool}. Los componentes clave de esta arquitectura incluyen:

\begin{itemize}
    \item \textbf{Proveedor de Blockchain (Provider)}: Interfaz que permite al frontend conectarse a la red blockchain. Puede ser un nodo local, un servicio de terceros como Infura \cite{infura}, o la wallet del usuario.
    \item \textbf{Wallet del Usuario}: Gestiona las claves privadas y firma las transacciones. Las wallets modernas como MetaMask inyectan un objeto \texttt{window.ethereum} en el navegador \cite{metamaskdocs}.
    \item \textbf{Librerías de Interacción}: Herramientas como Web3.js \cite{web3js}, Ethers.js \cite{ethersjs} o web3.py para interactuar con los contratos inteligentes.
    \item \textbf{ABI (Application Binary Interface)}: Define la interfaz del contrato inteligente, incluyendo sus funciones y eventos \cite{soliditydocs}.
\end{itemize}

\subsection{Proceso de Interacción}

El proceso típico para interactuar con un contrato inteligente desde el frontend incluye los siguientes pasos:

\begin{enumerate}
    \item \textbf{Detección del Proveedor}: El frontend detecta si el navegador tiene acceso a un proveedor de blockchain (por ejemplo, si MetaMask está instalado).

    \begin{lstlisting}
if (typeof window.ethereum !== 'undefined') {
    // Proveedor detectado
}
    \end{lstlisting}

    \item \textbf{Solicitud de Acceso a la Cuenta}: Se solicita permiso al usuario para acceder a su cuenta.

    \begin{lstlisting}
await window.ethereum.request({
    method: 'eth_requestAccounts'
});
    \end{lstlisting}

    \item \textbf{Creación de una Instancia del Contrato}: Se carga el ABI y se crea una instancia del contrato utilizando la dirección del contrato.

    \begin{lstlisting}
const contract = new ethers.Contract(
    contractAddress,
    abi,
    signer
);
    \end{lstlisting}

    \item \textbf{Llamadas a Funciones de Lectura}: Para obtener información sin modificar el estado, se realizan llamadas de tipo \texttt{call}.

    \begin{lstlisting}
const value = await contract.getValue();
    \end{lstlisting}

    \item \textbf{Transacciones para Modificar el Estado}: Para funciones que modifican el estado, se envían transacciones que deben ser firmadas por el usuario.

    \begin{lstlisting}
const tx = await contract.setValue(newValue);
await tx.wait();
    \end{lstlisting}

    \item \textbf{Manejo de Eventos}: Se pueden escuchar eventos emitidos por el contrato para actualizar la interfaz de usuario.

    \begin{lstlisting}
contract.on(
    'ValueChanged',
    (oldValue, newValue, event) => {
        // Actualizar UI
    }
);
    \end{lstlisting}

\end{enumerate}

\subsection{Consideraciones de Seguridad}

La interacción directa con contratos inteligentes desde el frontend requiere especial atención a la seguridad \cite{atzei2017survey}:

\begin{itemize}
    \item \textbf{Validación de Entradas}: Siempre validar las entradas del usuario antes de enviarlas al contrato para prevenir ataques como inyección de datos o overflows \cite{grishchenko2018semantic}.
    \item \textbf{Gestión de Errores}: Manejar adecuadamente las excepciones y errores tanto del lado del contrato como del frontend.
    \item \textbf{Protección Contra Reentrancy}: Aunque es más relevante en el contrato, el frontend debe ser consciente de las vulnerabilidades y diseñar interacciones seguras \cite{luu2016making}.
    \item \textbf{Uso de Canales Seguros}: Asegurarse de que la aplicación web se sirva a través de HTTPS para prevenir ataques man-in-the-middle \cite{wust2016ethereum}.
\end{itemize}

\subsection{Optimización de la Experiencia de Usuario}

Para mejorar la experiencia de usuario al interactuar con contratos inteligentes:

\begin{itemize}
    \item \textbf{Feedback en Tiempo Real}: Proporcionar indicadores visuales durante la espera de confirmación de transacciones.

    \item \textbf{Manejo de Gas Fees}: Informar al usuario sobre los costos de gas y permitir ajustar las tarifas \cite{zhou2020empirical}.

    \item \textbf{Compatibilidad Multinavegador y Multiplataforma}: Asegurar que la aplicación funcione correctamente en diferentes navegadores y con distintas wallets.

    \item \textbf{Internacionalización}: Soportar múltiples idiomas y formatos de moneda.

\end{itemize}

\subsection{Herramientas y Marcos de Trabajo}

Existen diversas herramientas y frameworks que facilitan el desarrollo y la interacción con contratos inteligentes:

\begin{itemize}
    \item \textbf{Truffle Suite}: Conjunto de herramientas para desarrollo, pruebas y despliegue de contratos inteligentes \cite{truffle}.

    \item \textbf{Hardhat}: Entorno de desarrollo para Ethereum que permite pruebas locales y depuración avanzada \cite{hardhat}.

    \item \textbf{OpenZeppelin}: Biblioteca de contratos inteligentes seguros y auditados \cite{openzeppelin}.

    \item \textbf{The Graph}: Protocolo para indexar y consultar datos de la blockchain de manera eficiente \cite{thegraph}.

\end{itemize}

\subsection{Casos de Uso}

Algunos casos de uso comunes de la interacción con contratos inteligentes desde el frontend incluyen:

\begin{itemize}
    \item \textbf{Aplicaciones DeFi}: Plataformas de finanzas descentralizadas como intercambios, préstamos y yield farming \cite{werner2021sok}.

    \item \textbf{NFT Marketplaces}: Mercados para la compra y venta de tokens no fungibles \cite{wang2021non}.

    \item \textbf{Juegos Blockchain}: Juegos que utilizan activos en la blockchain y permiten a los usuarios poseer y comerciar elementos del juego \cite{nfts2020}.

    \item \textbf{Sistemas de Votación Descentralizada}: Aplicaciones que permiten votaciones transparentes y seguras \cite{zhang2019survey}.

\end{itemize}

\subsection{Desafíos Técnicos}

A pesar de las ventajas, existen desafíos técnicos en la interacción con contratos inteligentes desde el frontend:

\begin{itemize}
    \item \textbf{Latencia y Escalabilidad}: Las transacciones en la blockchain pueden ser lentas y costosas. Soluciones como las sidechains y layer-2 ayudan a mitigar estos problemas \cite{croman2016scaling}.

    \item \textbf{Cambios en el Estado de la Red}: Las bifurcaciones (forks) y actualizaciones de la red pueden afectar la funcionalidad de las aplicaciones \cite{gudgeon2020defi}.

    \item \textbf{Compatibilidad con Múltiples Redes}: Soportar diferentes redes (Ethereum, Binance Smart Chain, Polygon) requiere manejo de configuraciones específicas \cite{wu2021defi}.

    \item \textbf{Seguridad de Contratos Inteligentes}: Bugs y vulnerabilidades en los contratos pueden llevar a pérdidas financieras. Auditorías y buenas prácticas de desarrollo son esenciales \cite{atzei2017survey}.

\end{itemize}

\subsection{Mejores Prácticas de Ingeniería}

Para desarrollar aplicaciones robustas que interactúan con contratos inteligentes:

\begin{itemize}
    \item \textbf{Modularidad}: Separar la lógica de negocio, la interacción con la blockchain y la interfaz de usuario.

    \item \textbf{Testing Extensivo}: Utilizar frameworks de prueba para contratos inteligentes y realizar pruebas unitarias y de integración \cite{mavridou2019verisolid}.

    \item \textbf{Documentación y Comentarios}: Mantener un código bien documentado facilita el mantenimiento y la colaboración.

    \item \textbf{Actualizaciones y Migraciones}: Planificar cómo manejar actualizaciones de contratos y migraciones de datos \cite{dannen2017introducing}.

\end{itemize}

\subsection{Futuras Tendencias}

El desarrollo en este campo está evolucionando rápidamente:

\begin{itemize}
    \item \textbf{Integración con WebAssembly (WASM)}: Posibilidad de escribir contratos en otros lenguajes y mejorar el rendimiento \cite{wood2019polkadot}.

    \item \textbf{Desarrollo de DApps Móviles}: Llevar la interacción con contratos inteligentes a dispositivos móviles \cite{he2018blockchain}.

    \item \textbf{Interoperabilidad entre Blockchains}: Protocolos que permiten la comunicación entre diferentes blockchains \cite{belchior2021survey}.

\end{itemize}
