\section{Descentralización y Control de Identidad}

La \textit{Identidad Descentralizada} (DID) es un concepto emergente que busca proporcionar a los individuos control total sobre sus identidades digitales. A diferencia de los sistemas tradicionales de identidad, que dependen de autoridades centrales como gobiernos o corporaciones, los DID se basan en tecnologías de blockchain para ofrecer identificadores únicos y verificables sin intermediarios.

Desde una perspectiva técnica, los DID utilizan criptografía de clave pública para asegurar que solo el propietario legítimo puede acceder y controlar su identidad. Los identificadores se almacenan en la blockchain, lo que garantiza su inmutabilidad y disponibilidad. Los documentos de identidad asociados a un DID incluyen información como claves públicas, métodos de autenticación y metadatos que permiten interacciones seguras.

En el ámbito de la ingeniería, la implementación de DID implica el desarrollo de protocolos y estándares que faciliten la interoperabilidad entre diferentes sistemas y plataformas. Organizaciones como el W3C han propuesto especificaciones para estandarizar los DID, permitiendo su integración en diversas aplicaciones.

La utilización de DID en DWebs permite a los usuarios autenticarse y acceder a servicios sin revelar información personal sensible, reduciendo el riesgo de violaciones de datos y ataques de phishing. Además, facilita la creación de sistemas de confianza distribuidos, donde la identidad y las credenciales pueden verificarse sin necesidad de una autoridad central.

La Figura~\ref{fig:DID_architecture} ilustra la arquitectura general de un sistema de Identidad Descentralizada, destacando los componentes clave y las interacciones entre ellos.

\begin{figure}[H]
    \centering
    \includegraphics[width=0.45\textwidth]{did_architecture.png}
    \caption{Arquitectura de un Sistema de Identidad Descentralizada}
    \label{fig:DID_architecture}
\end{figure}

Los desafíos técnicos incluyen la gestión de claves, la recuperación de identidades en caso de pérdida de claves privadas y la escalabilidad de los sistemas de blockchain utilizados. Soluciones como el uso de \textit{Hierarchical Deterministic Wallets} (HD Wallets) y esquemas de recuperación social están siendo exploradas para mitigar estos problemas.

