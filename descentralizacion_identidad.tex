\section{Descentralización y Control de Identidad}

La Identidad Descentralizada (DID) es un nuevo enfoque para la gestión de identidades que permite a los usuarios tener control total sobre sus datos personales \cite{w3cdid}. Las DID son identificadores únicos que se crean y controlan de forma autónoma, sin depender de autoridades centrales o proveedores de identidad \cite{preukschat2020self}. Esto se logra mediante el uso de criptografía de clave pública y tecnologías de contabilidad distribuida (DLT) \cite{allen2017decentralized}.

Desde una perspectiva técnica, una DID se representa como una URI (Uniform Resource Identifier) que puede resolver a documentos de identidad descentralizados (DID Documents) \cite{sporny2019did}. Estos documentos contienen las claves públicas y la información de servicio necesaria para interactuar con la identidad. La verificación de identidades se realiza mediante pruebas criptográficas, como firmas digitales y pruebas de conocimiento cero \cite{baldimtsi2013anonymous}.

La implementación de DID en las DWebs está cambiando la forma en que los usuarios acceden e interactúan con aplicaciones descentralizadas (dApps), eliminando la necesidad de intermediarios y reduciendo los riesgos de seguridad asociados con los sistemas centralizados \cite{tozny2017decentralized}. Por ejemplo, en lugar de utilizar nombres de usuario y contraseñas almacenados en servidores, los usuarios pueden autenticarse firmando desafíos criptográficos con sus claves privadas \cite{rivest1978method}.

Además, las DID permiten la interoperabilidad entre diferentes sistemas y plataformas, facilitando la portabilidad de la identidad y los datos asociados \cite{reed2016dkms}. Esto es especialmente relevante en entornos empresariales y gubernamentales, donde la gestión de identidades es crítica para la seguridad y el cumplimiento normativo \cite{cameron2005laws}.

Sin embargo, la adopción de DID plantea desafíos en términos de escalabilidad, estandarización y usabilidad. La resolución de DID puede ser costosa en términos computacionales y requiere infraestructura de soporte adecuada \cite{paul2018blockchain}. Asimismo, es necesario desarrollar interfaces de usuario intuitivas que permitan a los usuarios gestionar sus identidades de manera segura y sencilla.

