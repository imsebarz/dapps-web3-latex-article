\section{Autenticación a través de Wallets}

Las wallets digitales, como MetaMask y WalletConnect, han emergido como herramientas esenciales para la autenticación en aplicaciones descentralizadas \cite{metamask, walletconnect}. Estas wallets funcionan como gestores de claves privadas que permiten a los usuarios interactuar con la blockchain de forma segura. A diferencia de los métodos tradicionales de inicio de sesión que requieren nombres de usuario y contraseñas, las wallets permiten a los usuarios autenticarse mediante firmas criptográficas utilizando sus claves privadas.

Desde un punto de vista técnico, cuando un usuario desea autenticarse en una DWeb, el sitio web genera un mensaje aleatorio o un desafío que el usuario debe firmar con su clave privada \cite{hamilton2015ethereum}. Esta firma se verifica utilizando la clave pública asociada, garantizando que el usuario es el propietario legítimo de la dirección en la blockchain.

Este método de autenticación mejora significativamente la seguridad al eliminar la necesidad de almacenar credenciales sensibles en servidores centralizados, reduciendo el riesgo de brechas de seguridad y ataques como phishing y fuerza bruta. Además, las wallets ofrecen una experiencia de usuario más fluida, ya que permiten realizar transacciones y acceder a servicios con unos pocos clics desde el navegador.

La integración de wallets en las DWebs se realiza mediante APIs y protocolos estándar, como Ethereum JavaScript API (Web3.js) o Ethers.js, que facilitan la comunicación entre la aplicación web y la wallet del usuario \cite{web3js, ethersjs}. Estas librerías permiten a los desarrolladores solicitar firmas, enviar transacciones y acceder a la información de la blockchain de manera transparente.

Es importante destacar que las wallets también gestionan aspectos críticos como el manejo de gas fees, selección de redes (mainnet, testnets) y gestión de tokens, lo cual agrega complejidad técnica pero ofrece flexibilidad y control al usuario \cite{antonopoulos2018mastering}.

No obstante, este enfoque también presenta desafíos, como la dependencia de extensiones o aplicaciones externas, posibles vulnerabilidades en las wallets y la necesidad de educar a los usuarios sobre prácticas de seguridad en la gestión de sus claves privadas.

