\section{Autenticación a través de Wallets}

La autenticación tradicional en aplicaciones web generalmente involucra nombres de usuario y contraseñas, almacenadas y gestionadas por servidores centralizados. Este enfoque presenta múltiples vulnerabilidades, como el riesgo de ataques de fuerza bruta, phishing y violaciones de bases de datos. En el contexto de las DWebs, se ha adoptado el uso de wallets criptográficas como mecanismo de autenticación.

Las wallets, como MetaMask o WalletConnect, funcionan como interfaces que permiten a los usuarios interactuar con la blockchain. Desde una perspectiva técnica, estas wallets gestionan claves privadas que permiten firmar transacciones y autenticarse de forma segura. Al utilizar criptografía de clave pública, se asegura que solo el propietario de la clave privada puede realizar acciones en su nombre.

La Figura~\ref{fig:wallet_authentication} muestra el flujo de autenticación utilizando una wallet en una DWeb.

\begin{figure}[H]
    \centering
    \includegraphics[width=0.45\textwidth]{wallet_authentication.png}
    \caption{Flujo de Autenticación mediante Wallet en una DWeb}
    \label{fig:wallet_authentication}
\end{figure}

Los beneficios técnicos de este método incluyen:

\begin{itemize}
    \item \textbf{Seguridad Mejorada}: Al no transmitir ni almacenar contraseñas, se elimina un vector común de ataques.
    \item \textbf{Privacidad}: Los usuarios no necesitan proporcionar información personal para autenticarse.
    \item \textbf{Descentralización}: No depende de servidores centrales para la gestión de credenciales.
\end{itemize}

Desde el punto de vista de la ingeniería, la integración de wallets en aplicaciones web requiere el uso de APIs y bibliotecas específicas. Por ejemplo, MetaMask proporciona una API de JavaScript que permite a las aplicaciones web solicitar firmas de transacciones y acceso a la información de la cuenta.

Además, se utilizan estándares como EIP-4361 (Sign-In with Ethereum) que definen protocolos para la autenticación de usuarios mediante firmas criptográficas, asegurando interoperabilidad y buenas prácticas en el desarrollo.

Es importante considerar aspectos como la experiencia del usuario (UX), ya que la interacción con wallets puede ser compleja para usuarios no técnicos. Se deben diseñar interfaces intuitivas y educar a los usuarios sobre las prácticas de seguridad.

