\section{El Futuro del Control de Datos Personales en Web3}

La evolución de las tecnologías descentralizadas está encaminada a devolver el control de los datos personales a los usuarios \cite{w3cverifiable}. En el contexto de la Web3, se espera que las personas puedan gestionar su identidad y datos sin depender de terceros, promoviendo una internet más privada y segura.

\subsection{Identidad Auto-Soberana (SSI)}

La identidad auto-soberana es un modelo donde los individuos poseen y controlan su identidad digital sin intermediarios. Técnicamente, esto se logra mediante el uso de:

\begin{itemize}
    \item \textbf{Credenciales verificables}: Estándares que permiten emitir, compartir y verificar información de identidad de manera descentralizada \cite{w3cverifiable}.
    \item \textbf{Agentes de identidad}: Software que gestiona las identidades y credenciales del usuario \cite{preukschat2020self}.
    \item \textbf{Protocolos de comunicación seguros}: Para intercambiar información de forma confidencial y autenticada.
\end{itemize}

\subsection{Datos Personales como Activos}

Con tecnologías como los tokens no fungibles (NFTs), los datos personales pueden representarse como activos digitales controlados por el usuario. Esto permite monetizar y gestionar el acceso a la información personal de forma transparente.

\subsection{Computación Privada y Preservación de la Privacidad}

El desarrollo de técnicas como la computación multipartita segura (MPC), computación homomórfica y enclaves seguros permite procesar datos privados sin exponer la información subyacente. Estas tecnologías son fundamentales para aplicaciones en salud, finanzas y gobierno.

\subsection{Desafíos y Consideraciones}

A pesar del potencial, existen desafíos técnicos y regulatorios:

\begin{itemize}
    \item \textbf{Escalabilidad}: Implementar soluciones que puedan soportar una gran cantidad de usuarios y transacciones.
    \item \textbf{Interoperabilidad}: Asegurar que diferentes sistemas y redes puedan comunicarse y compartir datos de manera eficiente.
    \item \textbf{Regulaciones}: Cumplir con leyes de protección de datos y privacidad, como GDPR y CCPA \cite{finck2018blockchain}.
    \item \textbf{Adopción y Usabilidad}: Crear soluciones que sean atractivas y fáciles de usar para el público general.
\end{itemize}

El impacto de empoderar a los usuarios con mayor control sobre sus datos personales puede ser profundo, cambiando la forma en que interactuamos en línea y fomentando nuevas formas de innovación y colaboración. Los ingenieros y desarrolladores tienen la responsabilidad de crear sistemas que sean seguros, escalables y centrados en el usuario.

